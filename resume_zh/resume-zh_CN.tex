% !TEX TS-program = xelatex
% !TEX encoding = UTF-8 Unicode
% !Mode:: "TeX:UTF-8"

\documentclass{resume}
\usepackage{zh_CN-Adobefonts_external} % Simplified Chinese Support using external fonts (./fonts/zh_CN-Adobe/)
%\usepackage{zh_CN-Adobefonts_internal} % Simplified Chinese Support using system fonts
\usepackage{linespacing_fix} % disable extra space before next section
\usepackage{cite}

\begin{document}
\pagenumbering{gobble} % suppress displaying page number

\name{陈敏华}
\contactInfo{(+86) 152-2135-8935}{chenmhgo@gmail.com}{软件工程师}{github.com/chenminhua}

%%%%%%%%%%%%%%%%%%%%%%%%%%%%%% 工作经历 %%%%%%%%%%%%%%%%%%%%%%%%%%%%%%%
\section{工作经历}
\datedsubsection{\textbf{微软 | Microsoft}, Software Engineer 2}{2020.4-至今}
\begin{itemize}
  \item 负责提升数据库相关产品的部署方案优化,提升相关产品的可用性和用户体验。
  \item \textbf{主要技术栈}: Azure, Azure Stack, Powershell, CSharp.
\end{itemize}

\datedsubsection{\textbf{黑湖科技 | Blacklake}, 软件工程师}{2017.6-2020.4}
\datedsubsection{\textbf{黑湖科技 | Blacklake}, 实习软件工程师}{2016.11-2017.5}
\begin{itemize}
  \item 作为早期团队成员,完成黑湖智造整体后端架构从 0 到 1 的开发,独立设计并实现黑湖智造第一版的后端核心模块,并参与后端团队搭建工作。
  \item 参与黑湖智造 2.0 版本后端架构微服务化改造,并主导了黑湖智造生产执行与物料管理两大核心业务模块的架构设计。
  \item 设计并开发数据同步服务,并利用 elasticsearch 搭建搜索与用户信息追溯的后端组件。
  \item 设计并开发了可根据租户具体需求,配置不同对象存储的文件管理服务。
  \item 主导了数据历史版本追溯系统的性能优化,针对不同数据量采用自适应的数据模型,降低了 80\% 的磁盘占用和 30\% 的 api 延迟。
  \item 针对 BI 团队需求,主导开发了基于定时任务和基于消息队列的异步批量导出框架,降低了团队间沟通成本,并提升了应用负载的可预测性和可监控性。
  \item \textbf{主要技术栈}: Java, Kotlin, Spring Boot, MySQL, Kafka, Redis, Elastic Search, NodeJS, Kubernetes.
\end{itemize}

\datedsubsection{\textbf{IBM中国研究院},实习软件工程师}{2015.7-2016.1}
\begin{itemize}
  \item 负责多租户架构的 health care 服务原型开发。
  \item 基于 Node-RED 开发事件驱动的 emergency 报警。
  \item \textbf{主要技术栈}: Java, NodeJS, MySQL, RabbitMQ.
\end{itemize}

% \section{\faGraduationCap\ 教育背景}
\section{教育背景}
\datedsubsection{\textbf{中国科学院大学},通信与信息系统,\textit{工学硕士研究生}}{2014.8 - 2017.6}
\begin{itemize}
  \item \textbf{排名(前15\%)},核心课程: 算法导论,机器学习,计算机视觉,压缩感知,矩阵分析,机器人导论。
  \item 获得荣誉:全国研究生数学建模大赛三等奖 (2014.9)。
  \item 研究课题:Convolutional Recurrent Neural Network-based Channel Equalization.
\end{itemize}

\datedsubsection{\textbf{东南大学吴健雄学院(信息强化班)},通信工程,\textit{工学学士}}{2010.8 - 2014.6}
\begin{itemize}
  \item \textbf{排名 (前 30\%)}, 核心课程:数字电路与逻辑设计,模拟电路设计,信号与系统,数字信号处理。
  \item 获得荣誉:全国电子设计大赛江苏区二等奖 (2013.9),东南智能奖励金 (2013)。
\end{itemize}

% \section{\faCogs\ IT 技能}
\section{技术能力}
% increase linespacing [parsep=0.5ex]
\begin{itemize}[parsep=0.2ex]
  \item \textbf{编程语言}: Java (kotlin), JavaScript (Node.js), Golang, Python, C/C++, SQL 
  \item \textbf{常用工具与框架}: Mysql, Redis, Spring, Kubernetes, Tensorflow
  \item \textbf{其他关键词}: TCP/IP, Linux, 分布式系统。
\end{itemize}

\section{个人项目/作品摘要}
% increase linespacing [parsep=0.5ex]
\begin{itemize}[parsep=0.2ex]
  \item \textit{Blog},个人博客,https://chenminhua.github.io
  \item \textit{Suanfa},作为TA期间,为本科学生创建了一个关于算法和数据结构的网站 (2015 年),https://suanfa.herokuapp.com/0preface/
  \item \textit{Gituser},一个在命令行看 github user profile 的工具 (2016 年), https://github.com/chenminhua/gituser
  \item \textit{RNGIT},基于 RN 编写的移动端 github 客户端 (2016 年), https://github.com/chenminhua/RNGIT 
  \item \textit{Pin},一个跨网络云端文件剪贴板 (2019 年), https://github.com/chenminhua/pin

\end{itemize}

\section{自我评价}
\begin{itemize}[parsep=0.2ex]
  \item 喜欢阅读和思考,热爱科学,尤其喜欢研究数学。
  \item 有良好的英文听说读写能力,通过英语 6 级。 
  \item 目前比较擅长服务端架构设计与开发方面的工作,有一定的微服务架构经验。
\end{itemize}

%% Reference
%\newpage
%\bibliographystyle{IEEETran}
%\bibliography{mycite}
\end{document}
